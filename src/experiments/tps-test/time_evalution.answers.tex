\centering
\begin{longtable}[H]{|l|p{10cm}|}
\toprule
 & Answer \\
\midrule
Direct Ollama call & REANA is a reusable and reproducible analysis platform that supports various computational workflows using different types of computational backends such as Kubernetes, Slurm, or HTCondor. Below is a simple example of how you can create a "Hello World" workflow in REANA.

\#\#\# Step 1: Create the Workflow Specification

Firstly, you need to define your workflow specification file (e.g., `reana.yaml`). This file will describe the steps and tools required for your analysis.

Here's an example of what this might look like:

```yaml
specification:
  type: serial
  steps:
    - environment: 'python:3.8-slim'
      computational\_backend: kubernetes
      command: |
        python <<EOF
        print("Hello, World!")
        EOF
```

This workflow uses a Python 3.8 slim Docker image to run the script that prints "Hello, World!".

\#\#\# Step 2: Create Your Workflow

To create your REANA workflow, you need to have the `reana-client` installed and configured with access to a running REANA server. If not already done, install it using pip:

```bash
pip install reana-client
```

Then, configure the client by pointing it towards your REANA server (replace `<REANA\_SERVER\_URL>` with the actual URL of your REANA server):

```bash
export REANA\_SERVER\_URL=<REANA\_SERVER\_URL>
reana-client ping
```

\#\#\# Step 3: Upload and Run Your Workflow

Once you have your `reana.yaml` file ready, upload it to REANA:

```bash
reana-client create -n my-hello-world-workflow
reana-client upload reana.yaml
reana-client start my-hello-world-workflow
```

\#\#\# Step 4: Check the Results

You can check the status of your workflow and its output using the following commands:

To see the status:
```bash
reana-client status -w my-hello-world-workflow
```

To list all files in your workspace (including logs):
```bash
reana-client ls -w my-hello-world-workflow
```

And to download specific files, such as log files or output files:
```bash
reana-client download <filename> -w my-hello-world-workflow
```

This is a basic example of setting up and running a "Hello World" workflow in REANA. Depending on your needs, you can expand this with more complex workflows involving multiple steps, different computational backends, or even interactive sessions for data analysis. \\
Simple RAG call & Based on the context provided, I can help you draft a simple "Hello World" REANA workflow similar to the ones given in the examples.

Here's how your `helloworld.yaml` file might look:

```yaml
inputs:
  files:
    - helloworld.py

workflow:
  type: serial
  specification:
    steps:
      - environment: 'docker.io/library/python:3.10-bookworm'
        kubernetes\_memory\_limit: '100Mi'
        kubernetes\_job\_timeout: 60 \# seconds
        commands:
          - python helloworld.py
```

And your `helloworld.py` file should contain:

```python
print("Hello World!")
```

This workflow will run the Python script that prints "Hello World!" to the console. \\
FlowiseAI call & Pass Input:
```yaml
\# Hello World Workflow
inputs:
  files:
    - helloworld.py

workflow:
  type: serial
  specification:
    steps:
      - environment: 'docker.io/library/python:3.10-bookworm'
        kubernetes\_memory\_limit: '100Mi'
        kubernetes\_job\_timeout: 60 \# seconds
        commands:
          - python helloworld.py
```
```python
\# helloworld.py
print("Hello, World!")
``` \\
\bottomrule
\caption[]{Time consumption table - answers}
\end{longtable}

